%!TEX root = main.tex

\section{別ファイルのインクルード}

文章が長くなってくると,見通しが悪くなります.
そこで,ファイルを分割するのが有効です.
\index{ぶんかつ@分割}
今,表\ref{tab:filename}のようなファイルがあるものとします.

\begin{table}[h]
\caption{ファイル一覧}
\label{tab:filename}
\begin{center}
\begin{tabular}{l|l}
\hline
ファイル名 & 内容 \\ \hline
main.tex & 主となるファイル \\
input.tex & 取り込むファイル \\
\hline
\end{tabular}
\end{center}
\end{table}

\subsection{読み込む側の記述}
main.tex中に,input.texを取り込むための記述は以下のようになります.
読み込みたいファイルが複数有る場合は,その都度読み込んでください.
\index{main.tex}
\index{input.tex}

\begin{itembox}[c]{main.tex中の記述}
\begin{verbatim}
%!TEX root = main.tex

\section{別ファイルのインクルード}

文章が長くなってくると,見通しが悪くなります.
そこで,ファイルを分割するのが有効です.
\index{ぶんかつ@分割}
今,表\ref{tab:filename}のようなファイルがあるものとします.

\begin{table}[h]
\caption{ファイル一覧}
\label{tab:filename}
\begin{center}
\begin{tabular}{l|l}
\hline
ファイル名 & 内容 \\ \hline
main.tex & 主となるファイル \\
input.tex & 取り込むファイル \\
\hline
\end{tabular}
\end{center}
\end{table}

\subsection{読み込む側の記述}
main.tex中に,input.texを取り込むための記述は以下のようになります.
読み込みたいファイルが複数有る場合は,その都度読み込んでください.
\index{main.tex}
\index{input.tex}

\begin{itembox}[c]{main.tex中の記述}
\begin{verbatim}
%!TEX root = main.tex

\section{別ファイルのインクルード}

文章が長くなってくると,見通しが悪くなります.
そこで,ファイルを分割するのが有効です.
\index{ぶんかつ@分割}
今,表\ref{tab:filename}のようなファイルがあるものとします.

\begin{table}[h]
\caption{ファイル一覧}
\label{tab:filename}
\begin{center}
\begin{tabular}{l|l}
\hline
ファイル名 & 内容 \\ \hline
main.tex & 主となるファイル \\
input.tex & 取り込むファイル \\
\hline
\end{tabular}
\end{center}
\end{table}

\subsection{読み込む側の記述}
main.tex中に,input.texを取り込むための記述は以下のようになります.
読み込みたいファイルが複数有る場合は,その都度読み込んでください.
\index{main.tex}
\index{input.tex}

\begin{itembox}[c]{main.tex中の記述}
\begin{verbatim}
%!TEX root = main.tex

\section{別ファイルのインクルード}

文章が長くなってくると,見通しが悪くなります.
そこで,ファイルを分割するのが有効です.
\index{ぶんかつ@分割}
今,表\ref{tab:filename}のようなファイルがあるものとします.

\begin{table}[h]
\caption{ファイル一覧}
\label{tab:filename}
\begin{center}
\begin{tabular}{l|l}
\hline
ファイル名 & 内容 \\ \hline
main.tex & 主となるファイル \\
input.tex & 取り込むファイル \\
\hline
\end{tabular}
\end{center}
\end{table}

\subsection{読み込む側の記述}
main.tex中に,input.texを取り込むための記述は以下のようになります.
読み込みたいファイルが複数有る場合は,その都度読み込んでください.
\index{main.tex}
\index{input.tex}

\begin{itembox}[c]{main.tex中の記述}
\begin{verbatim}
\input{input}
\end{verbatim}
\end{itembox}

\subsection{読まれる側の記述}
input.tex中には,通常どおりの記述を行います.
ただし,1行目に以下の様な記述を行ってください.
これは,主となるファイルがmain.texであることを示しています.
これが書いてあると,TeXShopがメインとなるファイルを探し,関連するすべてのファイルを一気にコンパイルしてくれます.

もし,1行目にこの記述がない場合は,input.texをコンパイルしようとしてもbegin\{document\}が無い旨のエラーが出て止まってしまいます.
(その場合でも,main.texのウィンドウからタイプセットすればコンパイル可能ですが,面倒なのでキチンと書いておくことを勧めます)
\index{しゅとなるふぁいる@主となるファイル}
\index{main.tex}
なお,input.texに記述する内容は,1行目を除けばmain.texの一部がたまたま別ファイルになっているイメージです.
つまり,あらためてbegin\{doccument\}などを書く必要はありません.

\begin{itembox}[c]{input.texの1行目}
\begin{verbatim}
%!TEX root = main.tex
\end{verbatim}
\end{itembox}


\end{verbatim}
\end{itembox}

\subsection{読まれる側の記述}
input.tex中には,通常どおりの記述を行います.
ただし,1行目に以下の様な記述を行ってください.
これは,主となるファイルがmain.texであることを示しています.
これが書いてあると,TeXShopがメインとなるファイルを探し,関連するすべてのファイルを一気にコンパイルしてくれます.

もし,1行目にこの記述がない場合は,input.texをコンパイルしようとしてもbegin\{document\}が無い旨のエラーが出て止まってしまいます.
(その場合でも,main.texのウィンドウからタイプセットすればコンパイル可能ですが,面倒なのでキチンと書いておくことを勧めます)
\index{しゅとなるふぁいる@主となるファイル}
\index{main.tex}
なお,input.texに記述する内容は,1行目を除けばmain.texの一部がたまたま別ファイルになっているイメージです.
つまり,あらためてbegin\{doccument\}などを書く必要はありません.

\begin{itembox}[c]{input.texの1行目}
\begin{verbatim}
%!TEX root = main.tex
\end{verbatim}
\end{itembox}


\end{verbatim}
\end{itembox}

\subsection{読まれる側の記述}
input.tex中には,通常どおりの記述を行います.
ただし,1行目に以下の様な記述を行ってください.
これは,主となるファイルがmain.texであることを示しています.
これが書いてあると,TeXShopがメインとなるファイルを探し,関連するすべてのファイルを一気にコンパイルしてくれます.

もし,1行目にこの記述がない場合は,input.texをコンパイルしようとしてもbegin\{document\}が無い旨のエラーが出て止まってしまいます.
(その場合でも,main.texのウィンドウからタイプセットすればコンパイル可能ですが,面倒なのでキチンと書いておくことを勧めます)
\index{しゅとなるふぁいる@主となるファイル}
\index{main.tex}
なお,input.texに記述する内容は,1行目を除けばmain.texの一部がたまたま別ファイルになっているイメージです.
つまり,あらためてbegin\{doccument\}などを書く必要はありません.

\begin{itembox}[c]{input.texの1行目}
\begin{verbatim}
%!TEX root = main.tex
\end{verbatim}
\end{itembox}


\end{verbatim}
\end{itembox}

\subsection{読まれる側の記述}
input.tex中には,通常どおりの記述を行います.
ただし,1行目に以下の様な記述を行ってください.
これは,主となるファイルがmain.texであることを示しています.
これが書いてあると,TeXShopがメインとなるファイルを探し,関連するすべてのファイルを一気にコンパイルしてくれます.

もし,1行目にこの記述がない場合は,input.texをコンパイルしようとしてもbegin\{document\}が無い旨のエラーが出て止まってしまいます.
(その場合でも,main.texのウィンドウからタイプセットすればコンパイル可能ですが,面倒なのでキチンと書いておくことを勧めます)
\index{しゅとなるふぁいる@主となるファイル}
\index{main.tex}
なお,input.texに記述する内容は,1行目を除けばmain.texの一部がたまたま別ファイルになっているイメージです.
つまり,あらためてbegin\{doccument\}などを書く必要はありません.

\begin{itembox}[c]{input.texの1行目}
\begin{verbatim}
%!TEX root = main.tex
\end{verbatim}
\end{itembox}

