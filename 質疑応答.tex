% !TEX encoding = UTF-8 Unicode

\documentclass[a4j,10pt]{jsarticle}

\usepackage{newtxtext} % アルファベットなどをTimesNewRomanで
%\usepackage{newpxtext} % アルファベットなどをPalatinoで
% 両方共コメントアウトするとComputer Modernで

\usepackage{okumacro} %奥村先生による様々な便利機能
\usepackage{graphicx}
%\usepackage{url} %でき上がったPDFファイルからのリンク機能を有効にする・・・はず
%\usepackage{array}
\usepackage{tabularx} %2段組などで使う・・・はず
\usepackage{colortbl} %なんだっけ?
%\usepackage{float}
%\usepackage{makeidx} %目次を作ってくれる(ただしjsarticle用?)
%\makeindex
\usepackage{fouriernc}
\usepackage[deluxe]{otf} %日本語の太字をきちんと太字で表す
\usepackage{amsmath} %数式を,業界標準にする
\usepackage{moreverb} %ソースコードを引用する際に使う
%\usepackage{verbatim} %ソースコードを引用する際に使う(普段は上の行のを使う)
\usepackage[margin=2cm]{geometry} %上下左右の余白設定
\usepackage{dcolumn} 
\usepackage{ascmac}
%\usepackage{hyperref} %でき上がったPDFファイルからのリンク機能を有効にする・・・はず
\usepackage{color} %文字の色を変更する場合に使う
\usepackage{enumerate} %箇条書きを装飾する際に使う
 \usepackage{url}

\renewcommand{\headfont}{\bfseries} %見出しの英数字をゴシックにしない

\begin{document}
\begin{flushright}


肥田\ 雄也 %製作者
\end{flushright}

\begin{center}
\textbf{ \LARGE{質疑応答} }
\end{center}


\begin{table}[h]
\begin{tabular}{|l|l|}
\hline
質問項目&回答                       \\ \hline
動画は載せないのか                   & 抽出中は刻々と状況が変化し,じっくり動画を見る時間がないため,記載していません  \\ \hline
サーバーで動いているのか                & ローカル環境で動かしています.サーバーでも動くとは思いますが,検証はしていません           \\ \hline
写真は自分で撮ったのか                 & 一部お借りしている写真もありますが,レシピの写真は殆ど自身で撮影したものです \\ \hline
蒸し方は見れるのか                   & プアオーバーのレシピ閲覧画面,STEP2で閲覧できます                        \\ \hline
挽き方は確認できるのか                 & 各レシピのSTEP1に挽き具合が記載されています.手順については記載していません    \\ \hline
\end{tabular}
\end{table}

*音声認識機能は,外部サーバーにて処理を行うため,インターネット接続がない場合利用できません.

*音声認識機能は,GoogleChromeのみで使用可能です.音声認識ボタンを押したのち,ブラウザ側でマイクの使用を許可して下さい.


\end{document}